
% The \phantomsection command is needed to create a link to a place in the document that is not a
% figure, equation, table, section, subsection, chapter, etc.
% https://tex.stackexchange.com/questions/44088/when-do-i-need-to-invoke-phantomsection
\phantomsection

% ---
\chapter{\lang{Final Remarks}{Considerações Finais}} \label{cap:conclusion}
\phantomsection

	O trabalho desenvolvido nesse Trabalho de Conclusão de Curso, embora, não 
	tenha sido completamente finalizado, possibilitou uma análise sobre as técnicas
	de \textit{Checkpoint/Restore} para aplicações \textit{Stateful} no Kubernetes.
	Através da ferramenta CRIU, propusemos uma forma de implementar o
	\textit{Checkpoint/Restore}, que embora não tenha sido possível ser implementada
	no nosso Serviço, provou-se útil para realizar o salvamento do estado e posterior
	recuperação de uma aplicação. Já na implementação com técnicas de
	\textit{Event Sourcing} conseguimos realizar o Serviço de \textit{Checkpoint/Restore}
	no Kubernetes com transparência, provendo uma cópia utilizável da aplicação na
	tolerância a falhas a partir da reprojeção das requisições feitas a aplicação.
	
	O Serviço desenvolvido ainda possui algumas falhas, além de não conseguir
	implementar	o salvamento e recuperação com CRIU. Este só funciona com aplicações
	de estado que possuam uma réplica, mais réplicas geram estados incosistentes
	entre as aplicações. Também não houve muito foco na segurança do
	Serviço, já que temos acesso às chaves de comunicação com a API do Kubernetes
	em nosso Interceptador, qualquer um que tenha acesso ao Interceptador, por
	exemplo, através de uma exposição pela internet HTTP poderia obter acesso às
	chaves da API do Kubernetes e obter e alterar o estado do Kubernetes, expondo
	inclusive Secrets que devem funcionar como segredos no \textit{cluster}.

	A performance do nosso Serviço fica em evidência no Capítulo \ref{cap:analise:experimental},
	onde verificamos que a implementação necessita de melhorias na performance para
	que seja possível ser utilizada em aplicações em ambientes de produção. Como
	mostramos no capítulo existe uma evidência de que o armazenamento persistente
	através de um banco de dados SQL gera muita sobrecarga no sistema, precisamos
	trabalhar em cima deste ponto, com modificações unindo implementação com
	armazenamento em memória e \textit{Checkpoint} com CRIU. Esta perda de
	performance já era esperada no início do trabalho.

	Por fim, o trabalho cumpriu com o seu objetivo, que era implementar um Serviço
	de \textit{Checkpoint/Restore} para aplicações \textit{Stateful} no Kubernetes.
	Conseguimos acoplar uma aplicação ao nosso Serviço a partir da utilização de
	anotações no Deployment, permitindo que o usuário do Serviço não tenha que
	realizar configurações na sua aplicação, realizar \textit{Checkpoints} passivos
	com técnicas de \textit{Event Sourcing} e restaurar a aplicação no caso de uma
	falha de forma transparente, onde a aplicação é restaurada para seu último estado
	de funcionamento. Também conseguimos entregar uma parte clara de como integrar o
	\textit{Checkpoint} com CRIU no nosso Serviço através da utilização do buildah
	para construção das imagens de recuperação.

\section{Trabalhos futuros}

Como trabalhos futuros podemos listar alguns pontos que faltaram fechar netes trabalho.
O primeiro é a implementação com CRIU para o \textit{Checkpoint/Restore}, este trabalho
mostrou que é possível utilizar a arquitetura proposta para realizar o
\textit{Checkpoint/Restore}, mas não houve a finalização da implementação com CRIU,
porém listamos as partes e problemas que faltam para ser efetivada a implementação. Com
uma implementação com CRIU, podemos unir a implementação com técnicas de
\textit{Event Sourcing} para criar um serviço adaptativo que permita trocar o tipo de
restauração e definir o momento do início dos salvamentos de estado ativos através do
CRIU com base na análise da performance das duas implementações. Seguindo os trabalhos
de \cite{muller2022architecture}, onde nossa arquitetura foi inspirada. 

Neste trabalho em momento algum utilizamos técnica preditivas para detectar o momento
que a aplicação poderia falhar, então, em determinados momentos poderiamos ter uma
requisição interceptada pelo Interceptador e que não tenha sido enviada a aplicação
falhante, em \cite{tran2022proactive} houve uma implementação de predição, poderíamos
adicionar isto a nossa implementação para prover um Serviço mais ativo na recuperação,
que não espere a falha acontecer para agir.

Nosso trabalho não agiu com aplicações distribuídas geograficamente, isto mostra uma
nova perspectiva, pois, será necessário espalhar os conteúdos tanto do Interceptador
quanto de imagens geradas para outros nós geograficamente distribuídos o que gera
latência de rede e tempo de restauração diferente dependendo da localização geográfica
do nó que falha. No trabalho de \cite{vayghan2021kubernetes}, tivemos uma análise da
distribuição geográfica e como poderíamos fazer para diminuir a latência através da
distribuição dos dados armazenados persistentemente.

O trabalho também não cobre aplicações que tenham mais de uma réplica com estado da
mesma aplicação. Neste Serviço cobrimos apenas aplicações com uma réplica,
aplicações com diversas réplicas necessitam de coordenaçao entre os interceptadores
para acordar em uma ordem das requisições com algoritmos de consenso para resolver
estes problemas, aliado ao problema de aplicações geograficamente distribuídas
também temos o consenso entre estas os diferentes nós geograficamente distribuídos.

Por fim, é possível aproveitar ainda mais de conceitos de \textit{Event Sourcing}.
A aplicação alvo se instrumentada também com este padrão de projeto pode prover uma
forma de obtermos o estado do sistema dela e salvarmos temporariamente no nosso
Serviço. Esta cópia pode ser o estado de cada objeto armazenado no sistema, no caso
de um banco de dados chave/valor, e a partir da finalização de uma requisição poderiamos
salvar esta cópia no Administrador de Estados e excluir requisições anteriores a ela,
reprojetando apenas as requisições a partir desta cópia.
