%% intro.tex
%%
%% Copyright 2017 Evandro Coan
%% Copyright 2012-2016 by abnTeX2 group at http://www.abntex.net.br/
%%
%% This work may be distributed and/or modified under the
%% conditions of the LaTeX Project Public License, either version 1.3
%% of this license or (at your option) any later version.
%% The latest version of this license is in
%%   http://www.latex-project.org/lppl.txt
%% and version 1.3 or later is part of all distributions of LaTeX
%% version 2005/12/01 or later.
%%
%% This work has the LPPL maintenance status `maintained'.
%% The Current Maintainer of this work is the Evandro Coan.
%%
%% The last Maintainer of this work was the abnTeX2 team, led
%% by Lauro César Araujo. Further information are available on
%% https://www.abntex.net.br/
%%
%% This work consists of a bunch of files. But originally there ware 3 files
%% which are renamed as follows:
%% Renamed the `abntex2-modelo-include-comandos` to `chapters/chapter_1.tex`
%% Renamed the `abntex2-modelo-trabalho-academico.tex` to `chapters/intro.tex`
%% Renamed the `abntex2-modelo-references.bib` to `aftertext/modelo-ufsc-references.bib`
%%
%% This file was originally the main template file, however this main file was
%% split into several new files, which are respectively drastically changed,
%% except this files which contains most of the main documentation message.
%%

% ------------------------------------------------------------------------
% ------------------------------------------------------------------------
% abnTeX2: Modelo de Trabalho Academico (tese de doutorado, dissertacao de
% mestrado e trabalhos monograficos em geral) em conformidade com
% ABNT NBR 14724:2011: Informacao e documentacao - Trabalhos academicos -
% Apresentacao
% ------------------------------------------------------------------------
% ------------------------------------------------------------------------

% The \phantomsection command is needed to create a link to a place in the document that is not a
% figure, equation, table, section, subsection, chapter, etc.
% https://tex.stackexchange.com/questions/44088/when-do-i-need-to-invoke-phantomsection
\phantomsection

% Multiple-language document - babel - selectlanguage vs begin/end{otherlanguage}
% https://tex.stackexchange.com/questions/36526/multiple-language-document-babel-selectlanguage-vs-begin-endotherlanguage
\begin{otherlanguage*}{english}

% The \phantomsection command is needed to create a link to a place in the document that is not a
% figure, equation, table, section, subsection, chapter, etc.
% https://tex.stackexchange.com/questions/44088/when-do-i-need-to-invoke-phantomsection
\phantomsection

% Multiple-language document - babel - selectlanguage vs begin/end{otherlanguage}
% https://tex.stackexchange.com/questions/36526/multiple-language-document-babel-selectlanguage-vs-begin-endotherlanguage
\begin{otherlanguage*}{brazil}

% https://tex.stackexchange.com/questions/5076/is-it-possible-to-keep-my-translation-together-with-original-text

\chapter{\lang{Introduction}{Introdução}} \label{cap:introducao}

% What does [t] and [ht] mean?
% https://tex.stackexchange.com/questions/8652/what-does-t-and-ht-mean
%
% How can I get rid of the LaTeX warning: Float too large for page?
% https://tex.stackexchange.com/questions/36252/how-can-i-get-rid-of-the-latex-warning-float-too-large-for-page
%
% "warning: Text page X contains only floats" How to suppress this warning?
% https://tex.stackexchange.com/questions/223149/warning-text-page-x-contains-only-floats-how-to-suppress-this-warning
%
% Make a table span multiple pages
% https://tex.stackexchange.com/questions/26462/make-a-table-span-multiple-pages
%
% How to make the longtable to work with centering & caption on memoir class?
% https://tex.stackexchange.com/questions/386541/how-to-make-the-longtable-to-work-with-centering-caption-on-memoir-class
%
% How to fix this Package array Error: Only one column-spec allowed?
% https://tex.stackexchange.com/questions/367069/how-to-fix-this-package-array-error-only-one-column-spec-allowed
%
% How to auto adjust my last table column width, and why is there Underfull \vbox badness on this table?
% https://tex.stackexchange.com/questions/387238/how-to-auto-adjust-my-last-table-column-width-and-why-is-there-underfull-vbox/387251

Com o crescimento de técnicas de utilização de microsserviços e computação em
nuvem, se criou uma alta necessidade para a alta disponibilidade dos serviços
\cite{vayghan2021kubernetes}. A virtualização leve através da utilização de
contêineres, desenvolvidos em cima das funcionalidades do Linux, como \textit{cgroups}
e \textit{namespaces}, que possibilitam a limitação e priorização de recursos sem máquinas
virtuais e a isolação completa de processos, redes e arquivos, respectivamente,
possibilitou uma melhoria no processo de disponibilização destes microsserviços
na nuvem \cite{laadan2010linux}. Desta forma, para facilitar o processo de
gerenciamento e comunicação entre diversos contêineres de microsserviços na
nuvem, aplicações com este intuito foram criadas \cite{vayghan2021kubernetes},
chamadas de orquestradores de contêineres, entre elas a mais popular é
o Kubernetes\cite{kubernetes}.

Em geral, os orquestradores de contêineres proveem um grande ferramental para a
replicação, recuperação e manutenção de alta disponibilidade para os contêineres,
suprindo a maioria das necessidades que aplicações sem estado de memória interna,
do tipo \textit{Stateless}, possuem \cite{vayghan2021kubernetes}. Entretanto,
quando precisamos de aplicações que possuam estado em memória, aplicações
\textit{Stateful}, e, que este estado seja importante para as futuras execuções
da aplicação, falta suporte nos orquestradores de contêineres, principalmente em
adoção das práticas de \textit{Checkpoint/Restore} \cite{muller2022architecture}
para o melhoria do tempo da recuperação de contêineres com falhas. Este tipo de
técnica permite salvar o estado de memória em uma nova imagem para recuperá-la
mais tarde. Embora, orquestradores de contêineres possuam suporte para aplicações
com estado em armazenamento de dados persistente, como o StatefulSet no Kubernetes,
não existe suporte nativo para o \textit{Checkpoint} de estado em memória da
aplicação\cite{tran2022proactive}.

Desta forma, este trabalho desenvolve um serviço transparente para realizar o
\textit{Checkpoint/Restore} de aplicações \textit{Stateful} executando em
contêineres no Kubernetes. Este serviço possibilita a tolerância a falhas
das aplicações com estado, onde, quando uma falha ocorrer a aplicação será reiniciada
ao último ponto de execução antes da falha.

\section{Objetivos}

\subsection{Objetivo Geral}

Desenvolver um serviço transparente para realização de \textit{Checkpoint/Restore}
de aplicações de contêineres \textit{Stateful} que execute de maneira nativa no
orquestrador de contêineres Kubernetes.

\subsection{Objetivos Específicos}

\begin{itemize}
    \item Desenvolver um serviço transparente para \textit{Checkpoint/Restore}
    de aplicações \textit{Stateful} executando no orquestrador de contêineres Kubernetes com foco em memória volátil das aplicações.
    \item Coletar métricas da performance da solução implementada para garantir
    eficiência no \textit{Checkpoint/Restore}, comparando com execução sem a aplicação.
    \item Identificar e avaliar classes de problema em que a ferramenta não tem adequação para
    utilização.
\end{itemize}

\section{Organização do Trabalho}

O restante desse trabalho está organizado da seguinte maneira: o
\autoref{cap:fundamentacao:teorica} apresenta alguns conceitos teóricos e técnicos
de algumas ferramentas e paradigmas diretamente relacionados ao desenvolvimento
dessa monografia. Na sequência, o \autoref{cap:trabalhos:relacionados} apresenta 
trabalhos relacionados com o nosso tema e o que foi desenvolvido. O
\autoref{cap:servico} aborda a definição do nosso serviço, sua arquitetura e
quais tecnologias serão utilizadas para prover o \textit{Checkpoint/Restore}.
No \autoref{cap:implementacao:servico} apresentamos a implementação do serviço, como
ele foi feito, o que enfretamos de dificuldade na implementação, quais desafios foram
resolvidos, quais desafios não foram resolvidos, abordamos duas perspectivas
diferentes, a da implementação com técnicas de \textit{Event Sourcing} e outra com a
implementação com a utilização de CRIU. No \autoref{cap:analise:experimental} apresentamos
as análises experimentais que realizamos sobre o serviço, coletando métricas e entendendo
os resultados obtidos. Por fim, no \autoref{cap:conclusion} são apresentadas
as considerações finais reforçando os resultados obtidos, apresentando partes que não
conseguimos alcançar do trabalho e cobrindo possíveis trabalhos futuros.

\end{otherlanguage*}
