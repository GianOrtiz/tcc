
% The \phantomsection command is needed to create a link to a place in the document that is not a
% figure, equation, table, section, subsection, chapter, etc.
% https://tex.stackexchange.com/questions/44088/when-do-i-need-to-invoke-phantomsection
\phantomsection

% Multiple-language document - babel - selectlanguage vs begin/end{otherlanguage}
% https://tex.stackexchange.com/questions/36526/multiple-language-document-babel-selectlanguage-vs-begin-endotherlanguage
\begin{otherlanguage*}{english}

% The \phantomsection command is needed to create a link to a place in the document that is not a
% figure, equation, table, section, subsection, chapter, etc.
% https://tex.stackexchange.com/questions/44088/when-do-i-need-to-invoke-phantomsection
\phantomsection

% Multiple-language document - babel - selectlanguage vs begin/end{otherlanguage}
% https://tex.stackexchange.com/questions/36526/multiple-language-document-babel-selectlanguage-vs-begin-endotherlanguage
\begin{otherlanguage*}{brazil}

\chapter{\lang{Service Implementation}{Implementação do Serviço}} \label{cap:implementacao:servico}

Neste capítulo iremos trazer a experiência de implementação do Serviço para duas
diferentes implantações a primeira com CRIU e cri-o, que não conseguimos finalizar,
mas que possui importantes experiências e resolução de problemas a serem relatadas, e,
a segunda sendo a implantação utilizando técnicas de \textit{Event Sourcing} que foi
finalizada. Antes de apresentar as duas implementações iremos continuar apresentando
mais aplicações, bibliotecas e plataformas que utilizamos durante a implementação do
Serviço.

\section{Aplicações, bibliotecas e plataformas}

Para a implementação tanto do \textit{Checkpoint/Restore} com CRIU quanto com técnicas
de \textit{Event Sourcing} foi necessário a utilização de aplicações, bibliotecas e
plataformas que facilitaram o nosso desenvolvimento.

Um dos pontos iniciais para a implementação é o local para realizar a implantação do
Serviço. Inicialmente escolhemos utilizar a plataforma Emulab \cite{White+:osdi02}
que fornece máquinas virtuais sob demanda para computação distribuída. Embora, tudo
tenha funcionado bem com estas máquinas inicialmente, houveram dificuldades para
fazer funcionar o CRIU juntamente com o Kubernetes nas máquinas. Então, optamos por
trocar para outra solução mais comercial, o Google Cloud, nesta plataforma obtemos
uma máquina virtual permanente, desta forma, não era necessário refazer a configuração
da máquina todas as vezes. De modo, a economizar nos recursos nossos testes de
implantação foram feitos em \textit{cluster} Kubernetes de máquinas com 2vCPU e 4GB
de memória, que é suficiente para executar um \textit{cluster} de nó único com Kubernetes.

Para possibilitar a existência da aplicação no Kubernetes nós precisamos encaixar a
aplicação no fluxo de reconciliação de estado do Kubernetes através do conceito do Operator.
Para isto, criamos três Operators que juntos implementam partes dos componentes da
nossa arquitetura. O Interceptador é o único que não será implementado a partir do Operator.
Teremos o controlador de Deployment, que é responsável por verificar novos Deployments
com anotação de monitoramento \textit{crsc.io/checkpoint-restore: true}, que indica que deve
ser monitorado para recuperação pelo Serviço, este controlador irá adicionar no Deployment
o Interceptador como um \textit{sidecar container}, que permitirá interceptar as requisições
à aplicação alvo. O controlador de Pod, será o responsável por monitorar os Pods e
verificar quando houver a anotação \textit{crsc.io/checkpoint-restore: true} e o contêiner
alvo da aplicação sofrer um falha, ou seja, não estiver no estado de Pronto do Kubernetes,
ele ser agendado para uma recuperação. Por fim, temos um outro controlador não totalmente
implementado, ele é o responsável por verificar o intervalo de \textit{Checkpoint} da
aplicação através da anotação do Deployment \textit{crsc.io/checkpoint-interval}, sempre
que um novo Deployment monitorado for verificado, um recurso de Checkpoint é criado, e este
é monitorado pelo controlador do Checkpoint, quando se dá o momento do \textit{Checkpoint}
o controlador se comunica com o Interceptador e depois cria uma nova imagem a partir do
resultado do salvamento de estado do interceptador. Iremos cobrir com mais ênfase cada um
deles nas seções posteriores. Para construir todos este Operador, utilizamos o framework
operator-sdk, que permite realizar a criação de recursos de Kubernetes e dos controladores
de maneira ágil e simples, abstraindo as intereções com o kube-apiserver e criação dos
Custom Resource Definition.

Ainda no contexto do \textit{Checkpoint}, como comentamos no capítulo de trabalhos
relacionados em \cite{schmidttransparent}, podemos obter um \textit{Checkpoint} da
aplicação utilizando o kubelet e as funcionalidades já implementadas para o cri-o
através da API do Kubernetes. Entretanto, isso não nos gera uma imagem no formato
do Open Container Initiative, devemos criar essa imagem a partir de todos os
passos que fizeram no trabalho em \cite{schmidttransparent}. O caminho que optamos
por seguir é um pouco diferente e apresenta uma facilidade maior, na implementação
de \cite{schmidttransparent} temos o problema de dar manutenção na geração da imagem
sempre que uma mudança na interface ocorrer. Por isso, optamos por utilizar uma
biblioteca e aplicação que já fizesse esta parte a abstraisse o trabalho para nós, a
biblioteca que encontramos, que também funciona como uma aplicação de linha de comando,
é o buildah\cite{buildah}.

Para armazenamento persistente dos dados da aplicação, como as requisições, utilizamos
o PostgreSQL \cite{postgresql} como banco de dados persistente. Ele será utilizado para
armazenar as requisições persistentemente quando necessário, mas, não a aplicação pode
se comportar com as requisições em memória ou em armazenamento persistente, cada forma
possui seus prós e contras que abordaremos nas próximas seções.

\section{Configuração do Cluster}

Para realizar a configuração do \textit{cluster} Kubernetes foi necessário realizar
algumas configurações e instalação de pacotes que fizessem CRIU, Kubernetes e buildah
funcionarem. Para configurar um \textit{cluster} Kubernetes com todas as funcionalidades
necessárias utilizamos Ubuntu 20.04, em máquinas com processador Intel Broadwell de
arquitetura x86/64 com 2vCPU e 4GB de memória, a partir disso precisamos executar a
sequência de comandos no Código \ref{listing:setup-script}, neste código utilizamos
o cri-o na versão 1.25, que é a versão necessária para fornecer suporte para CRIU no
\textit{Checkpoint} com o Kubernetes.

\begin{lstlisting}[language=bash,caption={Comandos de configuração da máquina para CRIU e cri-o.},label={listing:setup-script}]
   sudo sh -c 'echo "deb https://download.opensuse.org/repositories/devel:/kubic:/libcontainers:/stable/xUbuntu_20.04/ /" > /etc/apt/sources.list.d/devel:kubic:libcontainers:stable.list'
	sudo sh -c 'echo "deb http://download.opensuse.org/repositories/devel:/kubic:/libcontainers:/stable:/cri-o:/1.25/xUbuntu_20.04/ /" > /etc/apt/sources.list.d/devel:kubic:libcontainers:stable:cri-o:$CRIO_VERSION.list'
	curl -L https://download.opensuse.org/repositories/devel:/kubic:/libcontainers:/stable:/cri-o:/1.25/xUbuntu_20.04/Release.key | sudo apt-key add -
	curl -L https://download.opensuse.org/repositories/devel:/kubic:/libcontainers:/stable/xUbuntu_20.04/Release.key | sudo apt-key add -
	sudo apt-get update
	sudo apt-get install -y criu cri-o cri-o-runc
\end{lstlisting}

Primeiro adicionamos os repositórios para instalação do cri-o, que será utilizado como
\textit{runtime} de contêineres para o Kubernetes, e instalação dele, no mesmo comando
de instalação, instalamos o pacote do CRIU. Neste momento devemos editar o arquivo de
configuração do cri-o para habilitar o CRIU para \textit{Checkpoint} em /etc/crio/crio.conf
sobre a seção [crio.runtime] os valores no Código \ref{listing:crio-conf}.

\begin{lstlisting}[language=plaintext,caption={Configuração a ser incluída no arquivo de configurações do cri-o.},label={listing:crio-conf}]
[crio.runtime]
enable_criu_support = true
drop_infra_ctr = false
\end{lstlisting}

Para o runc, que é o \textit{backend} de \textit{runtime} de contêineres que o cri-o
utilizada devemos editar para permitir que as funcionalidades do CRIU funcionem com as
configurações do Código \ref{listing:runc-conf} em /etc/criu/runc.conf.

\begin{lstlisting}[language=plaintext,caption={Configuração a ser incluída no arquivo de configurações do runc para o CRIU.},label={listing:runc-conf}]
tcp-close
skip-in-flight
manage-cgroups=ignore
\end{lstlisting}

Agora, precisaremos iniciar o serviço do cri-o com o comando \textbf{sudo systemctl start crio}
que irá habilitar o cri-o a funcionar na máquina.

A próxima parte requer que tenhamos buildah instalado e o componentes principais do nó
mestre do Kubernetes, o kubeadm e o kubelet, como também o kubectl para acessar o cluster
através da interface da API do Kubernetes. Os comandos para instalação destes pacotes e
componentes podem ser instalados através do Código \ref{listing:kubernetes-setup}. O comando
\textbf{sudo apt-mark hold} é necessário para travar as versões de instalação dos componentes
do Kubernetes nas suas versões em caso de atualização do sistema.

\begin{lstlisting}[language=bash,caption={Instalação dos pacotes necessários para Kubernetes e buildah.},label={listing:kubernetes-setup}]
sudo apt-get install -y apt-transport-https ca-certificates curl buildah make
sudo mkdir /etc/apt/keyrings
sudo sh -c "curl -fsSL https://pkgs.k8s.io/core:/stable:/v1.25/deb/Release.key | gpg --dearmor -o /etc/apt/keyrings/kubernetes-apt-keyring.gpg"
sudo sh -c 'echo "deb [signed-by=/etc/apt/keyrings/kubernetes-apt-keyring.gpg] https://pkgs.k8s.io/core:/stable:/v1.25/deb/ /" | tee /etc/apt/sources.list.d/kubernetes.list'
sudo apt-get update
sudo apt-get install -y kubelet kubeadm kubectl
sudo apt-mark hold kubelet kubeadm kubectl
\end{lstlisting}

Como o \text{Checkpoint} para contêineres no Kubernetes ainda é uma funcionalidade em alfa para
o Kubernetes ela não é ativada automaticamente para toda nova instância de \textit{cluster}
Kubernetes. Precisamos alterar o funcionamento do Kubernetes para utilizar especificamente essa
funcionalidade, para isso existem os chamados \textit{Feature Gates} do Kubernetes que permitem
nas versões mais novas do Kubernetes ativar funcionalidades ainda não lançadas. Para ativar a
funcionalidade de \textit{Container Checkpoint} devemos alterar o arquivo de configuração que
determina a execução do kubelet no systemd do Ubuntu, em
/usr/lib/systemd/system/kubelet.service.d/10-kubeadm.conf, este arquivo deve estar como no
Código \ref{listing:kubelet-conf}. A parte que ativa o \textit{Feature Gate} é
\textbf{Environment="KUBELET\_FEATURE\_GATES\_ARGS=--feature-gates=ContainerCheckpoint=true"}. Já
a parte \textbf{Environment="KUBELET\_EXTRA\_ARGS=--cgroup-driver=systemd"} configura o kubelet
para utilizar o cgroup systemd para executar, isto coloca o cri-o no mesmo cgroup do kubelet, 
que permite que o último utilize o primeiro, caso isso não seja feito o kubelet não teria acesso
ao cri-o.

\begin{lstlisting}[language=plaintext,caption={Configuração do kubelet para executar no systemd com Feature Flag de ContainerCheckpoint e cgroup do systemd.},label={listing:kubelet-conf}]
[Service]
Environment="KUBELET_KUBECONFIG_ARGS=--bootstrap-kubeconfig=/etc/kubernetes/bootstrap-kubelet.conf --kubeconfig=/etc/kubernetes/kubelet.conf"
Environment="KUBELET_CONFIG_ARGS=--config=/var/lib/kubelet/config.yaml"
Environment="KUBELET_EXTRA_ARGS=--cgroup-driver=systemd"
Environment="KUBELET_FEATURE_GATES_ARGS=--feature-gates=ContainerCheckpoint=true"
# This is a file that "kubeadm init" and "kubeadm join" generates at runtime, populating the KUBELET_KUBEADM_ARGS variable dynamically
EnvironmentFile=-/var/lib/kubelet/kubeadm-flags.env
# This is a file that the user can use for overrides of the kubelet args as a last resort. Preferably, the user should use
# the .NodeRegistration.KubeletExtraArgs object in the configuration files instead. KUBELET_EXTRA_ARGS should be sourced from this file.
EnvironmentFile=-/etc/default/kubelet
ExecStart=
ExecStart=/usr/bin/kubelet $KUBELET_KUBECONFIG_ARGS $KUBELET_CONFIG_ARGS $KUBELET_KUBEADM_ARGS $KUBELET_EXTRA_ARGS $KUBELET_FEATURE_GATES_ARGS
\end{lstlisting}

Por fim, iniciamos o \textit{cluster} Kubernetes através dos comando no Código
\ref{listing:kubernetes-conf}, desativamos o swap da máquina para não criar conflitos que podem
ocorrer no kubelet. Outra seção do Código \ref{listing:kubernetes-conf} é a configuração do
KUBECONFIG, que é a configuração que o kubectl utiliza para se comunicar com a API do Kubernetes.
Ao final do listing nós adicionamos um controlador de rede para os Pods do nosso sistema, qualquer
um pode ser utilizado, mas optamos por utilizar o Calico \cite{calico} pela facilidade em
instalação como vista na seção do Código \ref{listing:kubernetes-conf}

\begin{lstlisting}[language=bash,caption={Inicialização do Kubernetes, configuração de acesso para o kubectl e instalação do administrador de rede para Pods Calico.},label={listing:kubernetes-conf}]
# Start Kubernetes

sudo swapoff -a
sudo kubeadm init --pod-network-cidr=192.168.0.0/16 --ignore-preflight-errors='all'

# Init kubectl

mkdir -p $HOME/.kube
sudo cp -i /etc/kubernetes/admin.conf $HOME/.kube/config
sudo chown $(id -u):$(id -g) $HOME/.kube/config

# Add network manager. We are going to install Calico.

kubectl create -f https://raw.githubusercontent.com/projectcalico/calico/v3.26.1/manifests/tigera-operator.yaml
kubectl create -f https://raw.githubusercontent.com/projectcalico/calico/v3.26.1/manifests/custom-resources.yaml
\end{lstlisting}

A partir das configurações listadas nesta seção teremos um \textit{cluster} Kubernetes funcional
pronto para iniciar nossas implementações, que permitem a utilização de CRIU para o \textit{Checkpoint}
e de buildah para construção das imagens de recuperação.

\section{Interceptador}

O nosso Interceptador como descrito nas seções anteriores é implantado a partir de um
\textit{sidecar container}, ambas as versões do Serviço, a de \textit{Checkpoint/Restore}
a partir do CRIU e a do \textit{Checkpoint/Restore} a partir de técnicas de
\textit{Event Sourcing} utilizam o mesmo Interceptador que é a funcionalidade comum aos
dois. Para criá-lo utilizamos a linguagem de programação Go e
realizamos uma construção de sua imagem utilizando Docker, não é necessário utilizar
cri-o para constroir a imagem neste caso. A aplicação consiste de um servidor HTTP,
este servidor possui algumas funcionalidades expostas que serão chamadas de acordo com
a necessidade dos outros componentes. O Interceptador possui um estado interno que pode
estar em dois valores diferentes, ou em Ativo ou em Aguardo. No Ativo enviamos todas
as requisições interceptadas a aplicação alvo do Serviço, já no estado Aguardo nós
realizamos o armazenamento das requisições e esperamos o estado retornar a Ativo para
continuar a entrega das requisições. Este fluxo é mostrado no diagrama X. %TODO

Para realizar um \textit{Checkpoint}, expomos uma rota HTTP de \textbf{/checkpoint}, que
chama a API do Kubernetes para que o kubelet comunique com o cri-o para realizar um 
\textit{Checkpoint} da aplicação executando no contêiner dentro do Pod alvo utilizando
CRIU. De modo a realizar a comunicação com o kube-apiserver precisamos estar autenticados
com o Kubernetes, deste modo, criamos um Secret no Kubernetes, uma configuração de segredo
disponível na API do Kubernetes, com o certificado e key de assinatura para comunicação
com o kube-apiserver, estes arquivos ficam localizados no cluster em
/etc/kubernetes/pki/apiserver-kubelet-client.key e em
/etc/kubernetes/pki/apiserver-kubelet-client.crt, adicionamos estes dois arquivos em um
Secret como no Listing \ref{listing:kubelet-secret}. A partir daí basta chamarmos a API
do Kubernetes na rota <https://ip-do-cluster/checkpoint/namespace/pod/container>
utilizando o método HTTP POST, onde ip-do-cluster é o IP do \textit{cluster} Kubernetes,
é o \textit{namespace} do Kubernetes em que se encontra a aplicação do Kubernetes, pod é
o Pod que está executando a aplicação alvo e container o contêiner que está executando a
aplicação alvo. A partir daí, um \textit{Checkpoint} é salvo na máquina em
/var/lib/kubelet/checkpoints na forma de um arquivo compactado da forma
checkpoint-pod\_namespace-container-timestamp.tar, em que os valores são os mesmo,
exceto, \textit{timestamp}, que representa a string do momento em que o \textit{Checkpoint}
foi feito.

%TODO listing:kubelet-secret

Outra rota do Interceptador é a do estado em \textbf{/state}, através de uma chamada
HTTP ao método POST podemos alterar o estado com o parâmetro na \textit{query} state,
sendo ou Active, ou Waiting, que coloca o Interceptador nos estados, respectivamente,
de Ativou e Aguardo. Como já explicado anteriormente cada estado tem uma forma de agir
quando interceptamos uma requisição à aplicação alvo.

Por fim, nós temos a rota \textbf{/reproject} que é a reprojeção de todas as requisições
interceptadas pelo Interceptador e aceita pela aplicação alvo. Esta reprojeção permite
que o estado seja alcançado. O idela nesta reprojeção seria que ela pudesse ser feita a
partir de uma versão das requisições. Entretanto, não a fizemos porque não foi possível
finalizar a parte do \textit{Checkpoint/Restore} com CRIU que necessitaria desta parte.

\section{Checkpoint/Restore com CRIU}

Na implementação de \textit{Checkpoint/Restore} com CRIU o que devemos configurar são
os três controladores, o controlador dos Deployments, o controlador dos Pods e o
controlador do nosso recurso customizado Checkpoint. Nesta implementação temos três
passos para toda aplicação monitorada, criação e implantação de um manifesto do
Deployment da aplicação no Kubernetes. Posterior criação do recurso customizado
Checkpoint e monitoramento das falhas do contêiner monitorado e posterior recuperação.

\subsection{Controladores}

Todos os controladores do Kubernetes possuem um ciclo de reconciliação que deve
verificar o estado do \textit{cluster} e, em caso, de modificação do estado de algum
recurso que ele monitora deve agir de acordo com a alteração de estado para alcançar
um estado dos outros recursos de acordo com o que ele tem definido através dos manifestos
dos recursos.

\subsubsection{Controlador de Deployment}

O controlador do Deployment tem como seu recurso monitorado os Deployments do
\textit{cluster}. Sempre que uma aplicação é criada ela deve ser criada com anotações
no manifesto do Deploment, como \textbf{crsc.io/checkpoint-restore} e
\textbf{crsc.io/checkpoint-interval}, que são respectivamente, o valor booleano que
define se a aplicação é monitorada pelo Serviço ou não e o valor do intervalo de
\textit{Checkpoint} ativo. O diagram da Figura \ref{fig:deployment-controller-diagram}
define o funcionamento do controlador.

%TODO fig:deployment-controller-diagram

A partir da alteração do estado do \textit{cluster} pela criação de um novo Deployment
monitorado o controlador realiza a criação de um novo recurso de Checkpoint. Este novo
recurso possui um manifesto com os valores de intervalo de \textit{Checkpoint} e também
as informações da aplicação monitorada, como o nome do Deployment. Este recurso será 
monitorado pelo controlador de Checkpoint que irá executar as ações necessárias para
alcançar o estado definido pelo manifesto.

Este controlador também deve injetar o Interceptador como um \textit{sidecar container}
ao Deployment da aplicação alvo. Sempre que um Deployment novo é adicionado com a 
anotação \textbf{crsc.io/checkpoint-restore} o manifesto do Deployment é alterado
para adicionar o contêiner do Interceptador como um \textit{sidecar container}, um novo
ReplicaSet é, então, criado pelo Kubernetes que criará um novo Pod com o nosso
Interceptador. Ao mesmo tempo o controlador cria um novo Service para o Interceptador no
novo Deployment que permite que seja acessado o Interceptador através de todo o cluster,
o Kubernetes lida com a parte do DNS e das configurações de rede.

Ao final temos a adição no manifesto do Deployment do conteúdo do contêiner como visto no
Listing \ref{listing:sidecar-container}. Este contêiner tem montado na imagem um volume
que indica os arquivos de assinatura na comunicação com o kube-apiserver que criamos na
Seção de configuração do \textit{cluster}. Isto permite que o nosso Interceptador possa
se comunicar de forma segura e autenticada com o Kubernetes para solicitar um novo
\textit{Checkpoint} ao kubelet.

%TODO listing:sidecar-container

\subsubsection{Controlador de Pod}

Para implementar o nosso Administrador de Estado, implementamos parte dele no controlador
de Pod. O controlador tem uma adiministração mais passiva, através do ciclo de
reconciliação, quando um Pod apresenta um problema, como por exemplo, um dos contêineres
falhar, o controlador filtar os Pods que possuem a anotação
\textbf{crsc.io/checkpoint-restore}, adicionando a uma fila, que, posteriormente será
restaurada. A restauração aconteceria a partir da configuração de utilização de uma nova
imagem à aplicação monitorada, entretanto, não conseguimos criar a imagem de
\textit{Checkpoint} e esta parte não foi feita. Na Figura \ref{fig:restore-pod} temos o
diagrama de como funciona esta parte, obtemos o Pod falhante, encontramos o contêiner a
ser restaurado, configuramos o estado do Interceptador para Aguardo, alteramos a imagem
do contêiner monitorado para o \textit{Checkpoint} mais recente e, em seguida
reprojetamos as requisições feitas a partir daquele \textit{Checkpoint} à aplicação. Ao
findar e o contêinere estiver no seu estado de Pronto, alteramos o estado do Interceptador
para Ativo e findamos a restauração. Desta forma, temos um contêiner da aplicação com o
estado mais recente possível para aplicação e a restauração foi de forma transaparente ao
usuário.

%TODO fig:restore-pod

\subsubsection{Controlador de Checkpoint}

O controlador de Checkpoint realiza o ciclo de reconciliação para os recursos que nosso
Operador criou do tipo Checkpoint. O recurso Checkpoint possui o intervalo que é
passado através da anotação \textbf{crsc.io/checkpoint-interval} e o nome do Deployment
que está sendo monitorado. Através disso nosso controlador cria um Ticker, que irá
realizar o \textit{Checkpoint} a cada intervalo dado, para isso podemos realizar uma
chamada HTTP à rota \textbf{/checkpoint} do Interceptador, que irá gerar o salvamento do
estado da aplicação. Após isso, iriamos utilizar o buildah como biblioteca para gerar
uma nova imagem no formato OCI, que seria salva no nosso kubelet para posteriormente ser
utilizada em uma recuperação.

Como o este controlador também implementa uma parte do nosso Administrador de Estado da 
arquitetura geral, ele também seria o responsável por salvar informações de metadados
sobre a imagem de salvamento criada. Entretanto, não foi possível criar uma aplicação
que conseguisse utilizar as bibliotecas do buildah para criação da imagem, muitos
problemas foram encontrados ao se utilizar overlayfs como comunicação de disco e não
houve tempo para realizar no nosso Serviço a geração da imagem. Porém, conseguimos criar
uma imagem com buildah através da sequência de comandos no Listing
\ref{listing:buildah-build}, que podem ser refeitos através da biblioteca e agregados ao
Serviço.

%TODO listing:buildah-build

Desta forma, proveriamos um serviço transparente para monitorar as aplicações do tipo
\textit{stateful} no Kubernetes, realizando \textit{Checkpoints} periódicos que depois
seriam utilizados para recuperar a aplicação a um determinado estado sempre que houvessem
falhas, provendo tolerância a falhas, como visto em \cite{vayghan2021kubernetes}
\cite{muller2022architecture} \cite{oh2018stateful} e \cite{schmidttransparent}.

\section{Checkpoint/Restore com técnicas de Event Sourcing}

Na nossa implementação a partir de técnicas de Event Sourcing tinhamos o objetivo de
prover uma forma de agilizar a restauração da aplicação sem o problema do tempo que leva
para realizar o \textit{Checkpoint} da imagem. Já esperávamos que a performance para
aplicações com muitas requisições fosse pior, pois, necessitariamos reenviar todas as
requisições desde o início do estado de Pronto da aplicação. Mas, a partir dela
poderíamos unir com a técnica de \textit{Checkpoint} com CRIU para realizar ou uma ou
outra dependendo do estado atual do sistema, eliminando o tempo gasto em um
\textit{Checkpoint} no começo do estado de vida da aplicação.

\subsection{Controladores}

\subsubsection{Controlador de Deployment}

O controlador de Deployment para a implementação com técnicas de Event Sourcing é a mesma
que para a implementação com CRIU, exceto, que, não criamos o recurso de Checkpoint neste
caso, pois, não há necessidade. Devido ao fato de sempre estarmos salvando as requisições
interceptados pelo Interceptador, sempre temos a ordem de requisições que precisamos
executar para obter o mesmo estado na aplicação alvo. Não precisamos realizar um
\textit{Checkpoint} periódico como no outro caso, pois, o salvamento de estado esta nos
eventos que são gerados pelas requisições. Neste controlador, também adicionamos o 
\textit{sidecar container} ao Deployment e criamos o Service que redireciona requisições
ao Interceptador. O Interceptador tem sempre a mesma implementação, o que permite ser
utilizado em diferentes sistemas e diferentes contextos, já que tudo que ele provê é uma
API.

\subsubsection{Controlador de Pod}

O controlador de Pod para a implementação com técnicas de Event Sourcing funciona
diferente da implementação com CRIU. Pois, neste caso não precisamos alterar a imagem da
aplicação alvo, ela permanece a mesma sempre, tudo que precisamos realizar é a sequência
na Figura \ref{fig:pod-controller-event-sourcing}, identificar o contêiner falhante na 
aplicação monitorado, agendar um restauração, na restauração, alterar o estado do
Interceptador para Aguardo, realizando o cache das requisições que chegarem enquanto a
restauração ocorre, realizar a reprojeção de todas as requisições que já chegaram à
aplicação, isto irá colocar a aplicação no estado anterior a sua falha, após isto,
configuramos o estado do Interceptador para Ativo, isto fará o Interceptador continuar
a enviar as requisições do cache e novas para a aplicação alvo.

%TODO fig:pod-controller-event-sourcing

Desta forma, teriamos uma implementação que supostamente perderia performance dado
o número das requisições. Quanto mais requisições o sistema recebe, mais demorado 
fica a reprojeção de todas as requisições, iremos validar este ponto na próxima Seção.
Entretanto, vale notar que esta é uma implementação não tão abordada em outros trabalhos,
exceto em \cite{muller2022architecture} que utiliza o conceito até certo ponto para
replicar as requisições a partir do último ponto de salvamento anterior a recuperação
da aplicação, o que diminui a latência em se recuperar a aplicação.

\end{otherlanguage*}
