

\addtotextpreliminarycontent{\lang{Agradecimentos}{Agradecimentos}}

\begin{agradecimentos}

\lang
{
    Greetings.
}
{
    Os agradecimento principais deste trabalho de conclusão de curso são a minha
    família, principalmente aos meus pais, Laércio Aparecido Ortiz e Rosana
    Aparecida Gonçalves Ortiz, que dentro do possível proveram tudo que podiam
    para que eu trilhasse minha jornada até aqui, dentre muitos problemas que
    ocorreram pelo caminho e seu amor incondicional. Aos meus irmão, João Victor
    Ortiz e Luan Henrique Ortiz, que foram meus primeiros amigos e que carrego
    fortes lembranças de momentos bons nesta vida. Também quero agradecer a minha
    namorada, Beatriz Alves dos Santos, que não me deixou pensar em nenhum momento
    que eu era incapaz de terminar a graduação durante o tempo que esteve comigo,
    quero agradecer novamente ao meu pai que pediu para que eu não desistisse da
    mesma algum tempo atrás. Todos da minha família estiveram do meu lado este ano
    me apoiando de alguma forma e me alegrando sempre que puderam.

	Aos meus amigos de diversos grupos, em principal os BMs, os quais a UFSC me deu
	e que tornaram meus dias melhores com discussões banais e às vezes filosóficas e
	profundas, ao grupo de amigos Capivari Lighters, que embora não nos vemos com
	tanta frequência tenho certeza que foram essenciais para eu me tornar quem eu sou
	hoje e quem eu serei amanhã. Aos outros amigos que não se encontram na união destes
	dois também gostaria de agradecer pelos momentos que passamos durante o ano de
	2023 e me livraram de muita dor de cabeça através de risadas e bons momentos, como
	as periódicas idas à Lanchonete Quebra Gelo nas quintas feiras.
	
	Agradeço também aos meus professores durante toda a minha jornada de conhecimento 
	e formação. Estes me ensinaram muitas coisas que ao longo do tempo posso ter julgado
	inútil, mas que no final me serviram para muitas coisas dentro e fora das instituições
	de ensino. Mas, a maior coisa que fica dos ensinamentos foi o aprendizado em aprender.
}
\end{agradecimentos}


%Mesmo padrão da seção primária, porém sem indicativo numérico. Assim como: Dedicatória, Resumo, Abstract, Sumário, Listas, Referências, Apêndices e Anexos.
%
%
%Corpo do texto, fonte 10,5, justificado, recuo especial da primeira linha de 1 cm, espaçamento simples.
%
